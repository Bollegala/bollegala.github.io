% !TEX TS-program = xelatexmk

\documentclass[a4paper,11pt]{article}

%A Few Useful Packages
\usepackage{url}
\usepackage{marvosym}
\usepackage{fontspec} 					%for loading fonts
\usepackage{xunicode,xltxtra,url,parskip} 	%other packages for formatting
\RequirePackage{color,graphicx}
\usepackage[usenames,dvipsnames]{xcolor}
\usepackage[big]{layaureo} 				%better formatting of the A4 page
% an alternative to Layaureo can be ** \usepackage{fullpage} **
\usepackage{supertabular} 				%for Grades
\usepackage{titlesec}					%custom \section



%Setup hyperref package, and colours for links
\usepackage{hyperref}
\definecolor{linkcolour}{rgb}{0,0.2,0.6}
\hypersetup{colorlinks,breaklinks,urlcolor=linkcolour, linkcolor=linkcolour}

% Citeall
%\usepackage{biblatex}
%\usepackage{citeall}
\usepackage{cite}
%\usepackage{natbib}
\usepackage{multibib}

\newcites{pri}{Referred Journal Papers}
\newcites{sec}{Referred Conference Papers}

\newcounter{firstbib}
 
%FONTS
\defaultfontfeatures{Mapping=tex-text}
%\setmainfont[SmallCapsFont = Fontin SmallCaps]{Fontin}
%%% modified for Karol Kozioł for ShareLaTeX use
\setmainfont[
SmallCapsFont = Fontin-SmallCaps.otf,
BoldFont = Fontin-Bold.otf,
ItalicFont = Fontin-Italic.otf
]
{Fontin.otf}
%%%


\titleformat{\section}{\Large\scshape\raggedright}{}{0em}{}[\titlerule]
\titlespacing{\section}{0pt}{3pt}{3pt}
%Tweak a bit the top margin
%\addtolength{\voffset}{-1.3cm}


%-------------WATERMARK TEST [**not part of a CV**]---------------
\usepackage[absolute]{textpos}

\setlength{\TPHorizModule}{30mm}
\setlength{\TPVertModule}{\TPHorizModule}
\textblockorigin{2mm}{0.65\paperheight}
\setlength{\parindent}{0pt}

%--------------------BEGIN DOCUMENT----------------------
\begin{document}


%\pagestyle{empty} % non-numbered pages

\font\fb=''[cmr10]'' %for use with \LaTeX command

%--------------------TITLE-------------
\par{\centering
		{\Large Curriculum Vitae -- Danushka \textsc{Bollegala}
	}\bigskip\par}

%--------------------SECTIONS-----------------------------------
%Section: Personal Data
\section{Personal Details}

\begin{tabular}{rl}
    \textsc{Name:} & Prof.~Danushka Bollegala\\
    \textsc{Position:} & Professor in Natural Language Processing\\
    \textsc{Address:}   & Department of Computer Science, University of Liverpool, L69 3BX, UK. \\
    \textsc{Phone:}     & +44-1517954283\\
    \textsc{Email:}     & \href{mailto:danushka@liverpool.ac.uk}{danushka@liverpool.ac.uk} \\
    \textsc{Web:} & \href{http://danushka.net}{danushka.net} \\
    \textsc{Orcid ID:} & \url{https://orcid.org/0000-0003-4476-7003}\\
    \textsc{DoB:}  & 20th of April 1979
\end{tabular}

%Further/Higher education (Institutions, dates of attendance, qualifications obtained with dates, prizes or other special academic distinctions with dates). If an award was classified, the class should be indicated.

\section{Higher Education}
\begin{tabular}{r|p{10cm}}
\textsc{2007 -- 2009} & PhD Computer Science, The University of Tokyo, Japan.  \emph{Summa Cum Laude} \\
\textsc{2005 -- 2007} & M.Sc. Computer Science, The University of Tokyo, Japan. \emph{Summa Cum Laude} \\
\textsc{2001-- 2005} & B.Sc. Computer Science, The University of Tokyo, Japan. \emph{Summa Cum Laude} \\
\end{tabular}

\section{Honours and Awards}
\begin{enumerate}
\item EPSRC Top Peer Reviewer's Award 2017.
\item Best Journal Paper of the Year 2014-2015, Japanese Society for Artificial Intelligence.
\item IEEE Young Author Award 2011.
\item Best paper award at the 2011 Genetic and Evolutionary Computation (GECCO) Conference.
\item Best poster paper at the 2010 Pacific Rim International Conference on Artificial Intelligence (PRICAI).
\item Annual Conference Award for the Best Paper at 2010 Japanese Society for Artificial Intelligence (JSAI).
\item Dean’s Award for the Best Doctoral Thesis of the Year 2010, University of Tokyo.
\item Dean’s Award for the Best Masters Thesis of the Year 2007, University of Tokyo.
\item Dean’s Award for the Best Undergraduate Thesis of the Year 2005, University of Tokyo.
\end{enumerate}


%Any other relevant training and qualifications, including membership of professional bodies (if membership is graded, indicate the highest grade so far attained for each body with date of attainment).

\section{Memberships and Fellowships}
\begin{tabular}{r p{11cm}}
\textsc{2017.09-current} & Senior Fellow Cookpad PLC  UK. \\
\textsc{2010.04-current} & Visiting Research Fellow, National Institute of Informatics, Japan.\\
\textsc{2010.04-current} & Member IEEE Computer Society. \\
\textsc{2010.04-current} & Member Association for Computational Linguistics.\\
\textsc{2010.04-current} & Executive Member of the Japanese Society for Artificial Intelligence.\\
\textsc{2011.01-2012.12} & Visiting Research Fellow, School of Informatics, University of Sussex, UK.\\
\textsc{2011.04--2011.06} & Visiting Research Fellow, Department of Computer Science, University of Cambridge, UK. \\
\textsc{2009.10--2010.03} & Japan Society for Promotion of Science (JSPS), Post-doctoral Research Fellow, University of Sussex, UK. \\
\textsc{2007.04--2009.09} & Japan Society for Promotion of Science (JSPS), Doctoral Research Fellow, University of Tokyo, Japan.\\
\textsc{2000.04--2007.03} & Japan Ministry of Education Overseas Full Scholarship, University of Tokyo, Japan.
\end{tabular}


%Employment record (employer, job title, dates) up to and including present post (excluding vacation employment and similar jobs).

\section{Employment Record}
\begin{tabular}{r|p{11cm}}
 \emph{Current} & Professor, \textsc{University of Liverpool}, UK. \\\textsc{October 2018} \\\multicolumn{2}{c}{} \\
 \emph{Current} & Amazon Scholar, \textsc{Amazonl}, Cambridge, UK. \\\textsc{January 2019} \\\multicolumn{2}{c}{} \\
 \textsc{September 2013 -- December 2018} & Senior Lecturer, \textsc{University of Liverpool}, UK.  \\\multicolumn{2}{c}{} \\
 \textsc{April 2012 -- August 2013} & Senior Assistant Professor, (\emph{koshi})  \textsc{University of Tokyo}, Japan. \\\multicolumn{2}{c}{} \\
\textsc{April 2010 -- March 2012} & Assistant Professor, \textsc{University of Tokyo}, Japan. \\\multicolumn{2}{c}{} \\
\textsc{October 2009 -- March 2010} & JSPS Post-doctoral Research Fellow, \textsc{University of Sussex}, UK. \\\multicolumn{2}{c}{} \\
\textsc{April 2007 -- September 2010} & Japan Society for Promotion of Science (JSPS), Doctoral Research Fellow (DC1) \textsc{University of Tokyo}, Japan. \\\multicolumn{2}{c}{} \\
\textsc{April 2005 -- March 2007} & Research Assistant, \textsf{National Institute of Advanced Industrial Science and Technology} (AIST), Japan. \\\multicolumn{2}{c}{} \\
\textsc{June 2005 -- March 2010} & Research Consultant, \textsf{FAST a Microsoft Subsidiary} (former FAST Search \& Transfer), Norway.
\end{tabular}

%\section{Research Excellence}

% check Russell Group median grant income for CS and compare what I have obtained.
% Books written and their sales
% TED talks, public talks
% Reviewer, Area Chair duties (PC co-chairing)


On research front, he has several high quality publications accepted at top rated venues in his field such as PloSOne journal, knowledge base systems, AI & Law, and the European Conference on Machine Learning. Since the last Professional Development Review (in Aug 2016) he has published 10 papers. He has been awarded two large-scale research grants Fletchers Digital Legal Assistant KTP grant by Innovate UK (490k) as PI, and Maritime Surveillance by Defence Science and Technology Laboratory (750k) as Co-I. He is also a Co-I of an EPSRC grant application under review related to Global Health Challenges call. It is noteworthy that the Fletchers KTP is the largest KTP grant that the university has ever received from the Innovate UK.

Danushka continues to lead the Liverpool Natural Language Processing Group (https://sites.google.com/site/nlpliverpool/home) and the Text Analytics workgroup in the university. Moreover, he was successful in obtaining a CDT grant lead by Prof. Alex Singleton on (https://datacdt.org/meet-the-team/) and a PhD student to begin next year.
%Danushka will be giving a tutorial on relation representation at the International Joint Conferences on Natural Language Processing (IJCNLP) to be held in Taiwan in November. Moreover, is he planning to organise a workshop at Knowledge Representation Conference in 2018. 

Danushka was invited by the Royal Society to express his expert opinions at the BEIS committee consisting MPs and leading UK academics on how artificial intelligence will affect the future job markets and its impact on UK economy. The invitees for this special committee included renowned figures such as Ian Weight (Labour MP), Richard Fuller (Conservative MP), Prof. Sir Cary Cooper (Univ. of Manchester), Prof. Neil Lawrance (Univ. of Sheffield and Amazon). Danushka was the NLP expert in this committee and explained how language tools will shape the future job market. The discussions of this committee will be released as a report by the Royal Society.

Danushka primary supervises four PhD students and they have successfully proceeded to their second or third year of the PhDs. All those students have published research papers at international venues in natural language processing or machine learning. Danushka second supervises three students including an XJTLU PhD student.

On teaching, Danushka taught the COMP 527 Data Mining and Visualisation module for the fourth consecutive year. This module has attracted wide interest across the university enrolling not only computer science students but also management, economics and geography students. This year the number of registered students for this module was 50, making it the highest among all MSc modules taught in the department. Danushka has significantly revolutionized both the delivery and evaluation of data mining. In particular, a YouTube channel was created to cover further details related to the topics covered in the module. iPython notebooks were developed to assist data mining coding of the students. The final exam and the resit exam were significantly changed to reflect problem solving questions. This effort was also praised by the external examiner as demonstrated by his comment “excellent set of questions”. A series of mock examination questions were distributed to the students over the course of the module to prepare them for this new examination format. Given that data mining is a highly applied field of study it is important for the students to not only know the theoretical foundations in data mining but also be able to apply that theoretical knowledge to solve real-world problems. These changes were appreciated by students as demonstrated in the overall high (3.5 or above) ratings in the end of module questionnaire. 

Danushka continues to look after the undergraduate admissions as the admissions tutor. Given that the department will be replacing all existing undergraduate degree programmes with two new programmes (Computer Science BSc Hons, and Software Development with Computer Science) from entry 2018, Danushka’s role as the admissions tutor has been a vital and a busy one. Danushka has proposed and got approval for the entry criteria for the new programmes, and have worked towards getting those changes reflected consistently across UCAS and university web sites. Moreover, Danushka has represented the department on faculty-level committee discussing the significance of BTEC as an entry qualification for computer science. Danushka has analysed the performance of BTEC students on our modules and have made recommendations to the HoD. A new mathematics test will be introduced to BTEC applicants from entry 2019. This mathematics test will be set and evaluated by Danushka on behalf of the department.

As it can be clearly seen from the contributions Danushka already exceeds the expectations of a Senior Lecture and is operating at the level of a Reader. Given this, Danushka is planning to apply for the Readership promotions next year. He would like to request departmental support and guidance on this in the future. 

\section{Leadership}

% Royal Society Engagement
% Conference, workshop organisations
% Tutorials, Key notes, invited talks
% Journal editorial ships (new Japanese one, JSAI)

\section{Teaching Excellence}

% Innovative teaching methods (YouTube channel, live streaming, interactive ipython notebooks)
% Good ratings 4 and above all years
% Peer Review excellence
% External examiner comments on COMP 527. YouTube channel, flip teaching. Emphasis on problem solving, introducing the latest development in machine learning such as deep learning into the course. Using python libraries to give an hands on experience of the tools that are used in the data science industry.

\section{Practice Indicators}

% Consultancies (skwile, IQBlade, BrainPad, Alt+, Cognite)

\section{Fellowships and Awards}
\begin{tabular}{r p{11cm}}
\textsc{2017.05-current} & Chief Scientific Officer (CSO) Alt Ltd. Japan, specialising in Personalised Artificial Intelligence (PAI)\\
\textsc{2017.09-current} & Senior Fellow Cookpad PLC  UK, specialising in cooking recipe analysis\\
\textsc{2017.01-current} & Advisor BrainPad PLC Japan, specialising in deep learning applications \\
\textsc{2018.04-current} & Advisor LexSnap UK, specialising in Legal Chatbot systems\\
\textsc{2017.03-2017.06} & Advisor Skwile UK, specialising in financial risk prediction \\
\textsc{2011.04--2011.06} & Visiting Research Fellow, Department of Computer Science, University of Cambridge, UK. \\
\textsc{2009.10--2010.03} & Japan Society for Promotion of Science (JSPS), Post-doctoral Research Fellow, University of Sussex, UK. \\
\textsc{2007.04--2009.09} & Japan Society for Promotion of Science (JSPS), Doctoral Research Fellow, University of Tokyo, Japan.\\
\textsc{2000.04--2007.03} & Japan Ministry of Education Overseas Full Scholarship, University of Tokyo, Japan
\end{tabular}

%Teaching experience and activity (past and present courses taught, length of courses, novel aspects, class sizes, teaching load, details of course development etc.).

\section{Teaching Experience}

\begin{description}
\item[COMP 527: Data Mining and Visualisation]
I have been teaching COMP 527 continuously since 2014. It is a master-level module and a compulsory module for the MSc programme in Big Data and High Performance Computing. This module carries 15 credits and is taught in the second semester.
Since I have taken over this module, I have re-written the module specifications and have added more practical elements such as the introduction of Python-based programming course work and lab assignments. 
Initially, video recordings of the lectures were made available on a dedicated YouTube channel, and later when the university introduced the video streaming platform, the video captures from the lectures were made available at \url{stream.liv.ac.uk}. 
The exam structure was also changed to introduce more problem solving-type questions and this effort has been repeatedly praised by the external examiners.
This module is a popular choice among the computer science PGT cohort and the number of students taking this module has increased significantly over the year from 16 in 2014 to 48 in 2018. It is the module with the largest number of student registration among all PGT modules in the department.
The module is also taken by a large number of PhD students from various CDT programmes such as the Data Analytics and Society CDT,
Risk CDT and Physics CDT because data science has become an integrated component in many research fields not limiting to computer science.
The student feedback for COMP 527 this year was extremely positive and all questions in the student evaluation received a high average rating of 4.1 or above.

\item[COMP 212: Distributed Systems]
This is a second year 15 credit undergraduate model that is optional for all computer science students at the Department of Computer Science, University of Liverpool. The module covers both theoretical as well as practical aspects of distributed computing 
In addition to the written exam, there are two programming assignments that must be implemented using the Java programming language, testing the students' understanding of the algorithms in distributed systems.
I taught this module for three consecutive years during the period from 2014 to 2016.
On average, 40 students were registered for this module during that period. The student feedback for COMP 212 has been positive during that period. 
According to the departmental policy, undergraduate admission tutor is given a lower teaching load, and as a result I discontinued teaching COMP 212 from 2017.

\item[C Programming (University of Tokyo)]
This is a second year compulsory module for all students in the Department of Information and Communication Engineering at the University of Tokyo, Japan. Techniques for optimising programmes written for lower-level hardware devices are covered in this modules.
This is an intensive programming-oriented module with weekly assignments. On average 150 students take this module and it is the responsibility of the module coordinator to assess each submitted assignment and provide detailed feedback on a weekly basis. 
This module enabled me to gain valuable experience related to teaching large cohorts. 
In particular, I developed a machine learning-based automatic programme evaluation system that can automatically grade the student assignments and highlight common mistakes. This enabled students to submit their assignments many times as they wish before the deadline and obtain real-time feedback. This also encouraged students to submit the assignments well before the deadlines.
After the deadline has passed, I personally verified the mistakes detected by the tool and provided an annotated feedback to the students.
It would have been extremely difficult and time consuming to conduct this module without this innovation.

\end{description}

%Leadership, professional and collegial experience (past and present departmental, Faculty, University and outside, including Committee memberships and memberships of professional bodies, editorial boards, etc.).

\section{Leadership, Professional and Collegial Experience}
\begin{tabular}{r p{11cm}}
\textsc{2017.10-current} & Head of the Data Mining and Machine Learning (DMML) Research Group.\\
\textsc{2016.01-current} & Undergraduate admissions tutor, Department of Computer Science, University of Liverpool \\
\textsc{2013.09-2016.01} & Disability Support Officer, Department of Computer Science, University of Liverpool \\
\textsc{2014-current} & Leader Natural Language Processing Group, University of Liverpool\\
\textsc{2017-2018} & Member of the Advisory Committee, AI and Future Jobs, Royal Society of Science.\\
\textsc{2017-current} & Full member of Engineering and  Physical Science Research Council (EPSRC) Peer Review College \\
\textsc{2017-current} & Assessor for the Irish Research Council \\
\textsc{2013-2015} & Associate Editor of the Transactions of the Japanese Society for Artificial Intelligence\\
\textsc{2016-current} & Associate Editor of the Journal of Computational Social Sciences\\
\textsc{2014-current} & Evaluator of Research Grants for Xi'an Jiaotong-Liverpool University
\end{tabular}

\section{Organisation of Scientific Meetings}
\begin{tabular}{r p{13cm}}
\textsc{2018} & \textbf{Co-organiser} of the Knowledge Representation and Reasoning in Natural Languages (KRNL) Workshop at the 16th International Conference on the Principles of Knowledge Representation and Reasoning\\
\textsc{2017} & \textbf{Local organiser} for the17th Annual Meeting of the International Society of Pharmacovigilance (ISoP)\\
\textsc{2017} & \textbf{Program chair} of the Pharmacovigilance and Social Media Workshop at ISoP\\
\textsc{2012} & \textbf{Program chair} of the International Organised Sessions (IOS) at the 26th Annual Conference of Japanese Society for Artificial Intelligence
\end{tabular}


\section{Program Committees}
\begin{tabular}{l p{13cm}}
2019 & Area Chair of the Machine Learning Track at EMNLP-2019\\
2014 -- & Senior Programme Committee member for the International Joint Conference in Artificial Intelligence\\
2014-- & Senior Programme Committee member for the AAAI Conference on Artificial Intelligence\\
2014 & Information Extraction Area Chair for the International Conference on Computational Linguistics (COLING)\\
2010-- & PC member of ACL, EMNLP, NIPS, WWW, COLING, LREC and reviewer for JAIR, TKDE, TKDD, JMLR, TACL journals.
\end{tabular}


%Research experience and impact activity (contracts and grants awarded, with amount, date, awarding body and title, research students supervised, impact case studies etc.).

\section{Research Grants}

Total research income so far GBP 2,064,100.

\begin{enumerate}

\item \textbf{Procedural Natural Language Inference} (Cookpad), PI, (GBP 48k), 2017-2019.

\item \textbf{Legal Advisor Dialogue Engine} (LexSnap), PI, (GBP 12k), 2017-2018.

\item \textbf{Algorithm Design for Automatic Classification of Transactions into a Taxonomy} (Rosslyn Data Technologies), PI, (GBP 5k), 2017-2018.

%\item \textbf{Deep Learning of Procedural Natural Language Understanding}, PhD Studentship, University of Liverpool, PI (GBP 60k), 2017-2020.

\item \textbf{Track Analytics For Effective Triage of Wide Area}, Defence Science Technology Laboratory (DSTL), Co-I (GBP 243k), 2017-2019.

\item \textbf{Digital Legal Assistant}, Knowledge Transfer Partnership (Innovate UK), PI (GBP 492k), 2017-2020.

\item \textbf{WEB-RADR}: Recognising Adverse Drug Reactions, (European Commission) Innovative Medicine Initiative, Co-PI (GBP 471k), 2014-2017.

\item \textbf{I knew that relation from news}, Innovation Voucher Scheme, University of Liverpool, PI (GBP 5k), 2015-2016.

\item \textbf{The Revierview Law Contract Map Project}, Knowledge Transfer Partnership (Innovate UK), Co-PI (GBP 269k), 2015-2018.

\item \textbf{Resolving Relational Ambiguity between Entities on the Web}, Microsoft Research (MSR) CORE-9 Research Grant, PI, (GBP 10K), 2013–2015.

\item \textbf{Domain Adaptation for Semantic Relation Extraction}, Japanese Society for the Promotion of Science (JSPS), Research Grant for Young Researchers (B). PI (GBP 20K), 2012–2015.

\item \textbf{Cross-Language Relational Search}, Microsoft Research (MSR) CORE-7 Research Grant, PI, (GBP 20K), 2011-2012.

\item \textbf{Developing a Cross-Language Web Search Engine}, Information Technology Promotion Agency of Japan (IPA) grant for explorative software development (Mito Project), PI, (GBP 26.5K), 2010-2011.

\item \textbf{A Latent Relational Search Engine}, Google Research Award, Co-PI, (GBP 18.6K), 2010–2011.

\item \textbf{Developing a Relational Search Engine to Retrieve Semantic Relations between Entities}, Japanese Society for the Promotion of Science (JSPS) research grant, PI. (GBP 29.8K), 2010-2012.

\item \textbf{Research grant for overseas visiting scholars}, Global Centre of Excellence (GCOE), Japan. PI. (GBP 9.7K), April 2011–June 2011.

\item \textbf{Extracting Attributes for Entities using Web Data}, Global Centre of Excellence (GCOE), Japan. Sole PI. (GBP 7.5K), 2010-2011.

\item \textbf{Learning to Rank Entities on the Web}, Microsoft Research (MSR) CORE-6 Research Grant, Co-PI, (GBP 19.4K), 2010-2011.

\item \textbf{Using Web Mining to Provide Useful Information to Drivers}, Toyota InfoTechnology Centre, Co-PI, (GBP 29.8K) 2009-2012.

\item \textbf{Disambiguating Personal Names on the Web}, Japan Society for the Promotion of Science (JSPS) Research Grant PI. (GBP 29.8K), 2007-2009.

\item \textbf{Using Network Theory and Machine Learning to Structure and Represent Information Available on the Web}, Co-PI, (GBP 298K), 2009–2012.
\end{enumerate}

\section{Supervision of PhD Students}
Graduated PhD Students under my (co)-supervision:
\begin{enumerate}
\item Asif Hussain Khan, graduated March 2014, now assistant professor, University of Dhaka.
\item Leon Palafox, graduated March 2012, now postdoc at University of Arizona.
\item Liu Shu, graduated March 2011, now engineer at Microsoft.
\item Makoto Tanji, graduated March 2011, now engineer at Wantedly.
\item Akio Watanabe, graduated March 2012, now engineer at CyberAgent.
\item Nguyen Tuan Duc, graduated March 2011, now engineer at Alt+.
\item Hugo Hernault, graduated March 2011, now engineer at Barclays.
\item  Abdullah Alsheri, graduated June 2017, now lecturer at Saudi
\end{enumerate}

\section{PhD Examinations:}
\begin{enumerate}
\item Mohammed Al-Zeyadi, University of Liverpool, July 2018.
\item Bastian Broecker, University of Liverpool, April 2018.
\item Fatima Abdullahi, University of Liverpool, May 2016.
\item Liyung Gong, University of Liverpool, November, 2014.
\item Tacoa Renevey Francisco, University of Tokyo, March, 2013.
\item Mamdouh Farouk Mohamed, University of Tokyo, March, 2012.
\item Haibo Li, University of Tokyo, March 2011.
\item Alena Neviarouskaya, University of Tokyo, March 2011.
\end{enumerate}

\section{My current PhD students:}
Pavithra Rajendran, Huda Hakami, Alsuhaibani Mohammed, Xia Cui, James O'Neill, Jodie Chou, Michael Abaho

%Invitations to speak (at conferences, international meetings, etc.).

%Industrial collaboration (consultancies, patents, and impact activity).



\section{Selected Keynotes/Invited Talks}
\begin{enumerate}
\item Keynote speech at National Human Resource Conference, Colombo, 2018.
\item Invited talk at International Society for Pharmacovigilance, 2017.
\item Invited talk at Microsoft Research Beijing Lab, 2013.
\item Keynote at Information-Based Induction Sciences (IBIS) Conference, 2011.
\item Invited talk at Google Mountain View Lab, 2011.
\item Invited talk at Microsoft Research Seattle Lab, 2010.
\item Keynote at First Japanese Web Symposium, 2009.
\end{enumerate}


\section{Patents}
\begin{enumerate}
\item Query Annonymisation via Semantic Decomposition, Japanese patent (filed 2018.06 and patent-pending).
\item A Method for Extracting the Semantic Relations that exist between two Entities from a Text Corpus, Japanese patent no: 2010-096551, 2010.
\item A Relational Search System, Japanese patent no: 2009-275762, 2009.
\end{enumerate}

\section{Faculty Impact case Records}
\begin{enumerate}
\item AI \& Law Impact case covering KTP projects with Riverview Law, Fletchers solicitors, and consultancy project with LexSnap.
\item Impact case for the monitoring adverse reactions of drugs from social media for pharmacovigilance (WEB-RADR project)
\end{enumerate}

\section{Industrial Collaborations/Consultancies}

\begin{tabular}{r p{11cm}}
%\textsc{2018.10-current} & Amazon Scholar\\
\textsc{2018.04-current} & NLP consultant, LexSnap Ltd.\\
\textsc{2017.08-current} & Senior Fellow, Cookpad Ltd. \\
\textsc{2017.11-current} & Chief Scientific Officer (CSO), Alt \\
\end{tabular}


\newpage
\section{Publications}

I have published over 125 papers in top international venues in Natural Language Processing, Machine Learning, Data Mining, Artificial Intelligence, and Social Media Analysis.My papers have been cited 2875 times with an $h$-index of 27 and i10-index of 49.
For a full list of my publications and metrics see \href{https://goo.gl/mnhgp6}{Google Scholar Profile}.

\bibliographystylepri{plainyr-rev}
\bibliographypri{journals}
\nocitepri{*}

\bibliographystylesec{plainyr-rev}
\bibliographysec{confs}
\nocitesec{*}

%\section{References}
%References are available upon request.
%
\end{document}
