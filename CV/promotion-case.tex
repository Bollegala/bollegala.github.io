\section{Research Excellence}

% check Russell Group median grant income for CS and compare what I have obtained.
% Books written and their sales
% TED talks, public talks
% Reviewer, Area Chair duties (PC co-chairing)


On research front, he has several high quality publications accepted at top rated venues in his field such as PloSOne journal, knowledge base systems, AI & Law, and the European Conference on Machine Learning. Since the last Professional Development Review (in Aug 2016) he has published 10 papers. He has been awarded two large-scale research grants Fletchers Digital Legal Assistant KTP grant by Innovate UK (490k) as PI, and Maritime Surveillance by Defence Science and Technology Laboratory (750k) as Co-I. He is also a Co-I of an EPSRC grant application under review related to Global Health Challenges call. It is noteworthy that the Fletchers KTP is the largest KTP grant that the university has ever received from the Innovate UK.

Danushka continues to lead the Liverpool Natural Language Processing Group (https://sites.google.com/site/nlpliverpool/home) and the Text Analytics workgroup in the university. Moreover, he was successful in obtaining a CDT grant lead by Prof. Alex Singleton on (https://datacdt.org/meet-the-team/) and a PhD student to begin next year.
%Danushka will be giving a tutorial on relation representation at the International Joint Conferences on Natural Language Processing (IJCNLP) to be held in Taiwan in November. Moreover, is he planning to organise a workshop at Knowledge Representation Conference in 2018. 

Danushka was invited by the Royal Society to express his expert opinions at the BEIS committee consisting MPs and leading UK academics on how artificial intelligence will affect the future job markets and its impact on UK economy. The invitees for this special committee included renowned figures such as Ian Weight (Labour MP), Richard Fuller (Conservative MP), Prof. Sir Cary Cooper (Univ. of Manchester), Prof. Neil Lawrance (Univ. of Sheffield and Amazon). Danushka was the NLP expert in this committee and explained how language tools will shape the future job market. The discussions of this committee will be released as a report by the Royal Society.

Danushka primary supervises four PhD students and they have successfully proceeded to their second or third year of the PhDs. All those students have published research papers at international venues in natural language processing or machine learning. Danushka second supervises three students including an XJTLU PhD student.

On teaching, Danushka taught the COMP 527 Data Mining and Visualisation module for the fourth consecutive year. This module has attracted wide interest across the university enrolling not only computer science students but also management, economics and geography students. This year the number of registered students for this module was 50, making it the highest among all MSc modules taught in the department. Danushka has significantly revolutionized both the delivery and evaluation of data mining. In particular, a YouTube channel was created to cover further details related to the topics covered in the module. iPython notebooks were developed to assist data mining coding of the students. The final exam and the resit exam were significantly changed to reflect problem solving questions. This effort was also praised by the external examiner as demonstrated by his comment “excellent set of questions”. A series of mock examination questions were distributed to the students over the course of the module to prepare them for this new examination format. Given that data mining is a highly applied field of study it is important for the students to not only know the theoretical foundations in data mining but also be able to apply that theoretical knowledge to solve real-world problems. These changes were appreciated by students as demonstrated in the overall high (3.5 or above) ratings in the end of module questionnaire. 

Danushka continues to look after the undergraduate admissions as the admissions tutor. Given that the department will be replacing all existing undergraduate degree programmes with two new programmes (Computer Science BSc Hons, and Software Development with Computer Science) from entry 2018, Danushka’s role as the admissions tutor has been a vital and a busy one. Danushka has proposed and got approval for the entry criteria for the new programmes, and have worked towards getting those changes reflected consistently across UCAS and university web sites. Moreover, Danushka has represented the department on faculty-level committee discussing the significance of BTEC as an entry qualification for computer science. Danushka has analysed the performance of BTEC students on our modules and have made recommendations to the HoD. A new mathematics test will be introduced to BTEC applicants from entry 2019. This mathematics test will be set and evaluated by Danushka on behalf of the department.

As it can be clearly seen from the contributions Danushka already exceeds the expectations of a Senior Lecture and is operating at the level of a Reader. Given this, Danushka is planning to apply for the Readership promotions next year. He would like to request departmental support and guidance on this in the future. 

\section{Leadership}

% Royal Society Engagement
% Conference, workshop organisations
% Tutorials, Key notes, invited talks
% Journal editorial ships (new Japanese one, JSAI)

\section{Teaching Excellence}

% Innovative teaching methods (YouTube channel, live streaming, interactive ipython notebooks)
% Good ratings 4 and above all years
% Peer Review excellence
% External examiner comments on COMP 527. YouTube channel, flip teaching. Emphasis on problem solving, introducing the latest development in machine learning such as deep learning into the course. Using python libraries to give an hands on experience of the tools that are used in the data science industry.

\section{Practice Indicators}

% Consultancies (skwile, IQBlade, BrainPad, Alt+, Cognite)